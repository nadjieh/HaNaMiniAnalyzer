\section{Strategy for Bunch-by-Bunch studies}
\begin{frame}
  \frametitle{Bunch-by-Bunch studies}
  \begin{itemize}
  \item With all respects to the LUM POG, how much should we care about the source of the time evolution?
    \begin{itemize}
    \item As Georgios suggested, we need to make sure that this is a real effect
    \item It needs to be cross checked with another luminometer
    \end{itemize}
  \item This study has some motivations for Run II
  \item To conherently add this to the Run II PU study
    \begin{itemize}
    \item we can study the effect of considering it in Run II
    \item a systematic uncertainty can be assigned to the best fit cross section for the assumption of the bunch-by-bunch luminosity profile
      \begin{itemize}
      \item baseline: Gaussian
      \item compared to other scenarios
      \end{itemize}
    \end{itemize}
  \end{itemize}
\end{frame}
\begin{frame}
  \frametitle{What do we need to do}
  \begin{itemize}
  \item to estimate the shape of the bunch-by-bunch luminosity
    \begin{itemize}
    \item divide the fill in different time intervals (30 minutes or so)
    \item fit to find a function to describe it (multiple function forms can be used here)
    \item the final shape of the fill is sum of these functions
      \begin{equation}
        f = \sum_{i}Lumi_{i}\times f_{i}
      \end{equation}
    \end{itemize}
  \item this function can be used as the input of puCalculator
    \begin{itemize}
      \item do we know how to implement it?
    \end{itemize}
  \item the output can be used to estimate the pp inelastic cross section
  \item comparing with nominal values, the uncertainty can be assigned
  \item for the sake of simplicity, we can start with one long fill instead of entire Run-II
  \end{itemize}
\end{frame}
